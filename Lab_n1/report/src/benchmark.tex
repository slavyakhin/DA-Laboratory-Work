\section{Тест производительности}

Тест производительности представляет из себя следующее: поразрядная сортировка сравнивается с $std::stable\_sort()$, чья сложность, согласно стандарту C++17, $O(n*\log(n))/O(n*(\log(n))^{2})$ (В GCC реализация --- merge sort).
Тестовый набор данных содержит в себе 1000000 ($10^{6}$) строк --- пар ключ-значение.

\begin{alltt}
root@77eb62ed9309:/workspaces/DA-Laboratory-Work/Lab_n1/solution# g++ -std=c++17 -Wall -Wextra -I. -o radix main.cpp       
root@77eb62ed9309:/workspaces/DA-Laboratory-Work/Lab_n1/solution# ./radix
RadixSort time: 1.12129
root@77eb62ed9309:/workspaces/DA-Laboratory-Work/Lab_n1/solution# g++ -std=c++17 -Wall -Wextra -I. -o std_sort std_sort.cpp 
root@77eb62ed9309:/workspaces/DA-Laboratory-Work/Lab_n1/solution# ./std_sort 
std::stable_sort time: 2.21333
\end{alltt}

Как видим, поразрядная сортировка оказалась значительно боллее эффективным методом для такого набора данных.

\pagebreak

