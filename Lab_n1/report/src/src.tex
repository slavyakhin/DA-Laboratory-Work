\section{Описание}
Требуется написать реализацию алгоритма поразрядной сортировки.

Поразрядная сортировка используется для наборов данных, в которых ключ можно поделить на "разряды" такие как,
например, цифры в десятичном числе или небольшие группы бит в двоичном числе.
Если такие разряды могут обладают достаточно небольшим набором значений, то для каждого из них удобно использовать сортировку подсчетом.
Таким образом, применив сортировку подсчетом для каждого разряда можно отсортировать набор данных.

Тогда временная сложность алгоритма будет $O(n\omega)$, где

n --- количество сортируемых элементов

$\omega$ --- количество разрядов в одном ключе

\pagebreak

\section{Исходный код}
Программа состоит из нескольких файлов.

На каждой непустой строке входного файла располагается пара \enquote{ключ-значение}, поэтому создадим новую 
структуру $TKeyValuePair$, в которой будем хранить ключ и значение.
Для хранения ключа тоже создадим структуру $TDate$ в которой реализуем поразрядное представление даты.
Также напишем свою реализацию динамического массива $TVector$ для хранения данных.

\begin{longtable}{|p{7.5cm}|p{7.5cm}|}
\hline
\rowcolor{lightgray}
\multicolumn{2}{|c|} {main.cpp}\\
\hline
int main()&Точка входа в программу. Объявление контейнера, ввод данных, вызов сортировки, вывод отсортированного массива\\
\hline
\rowcolor{lightgray}
\multicolumn{2}{|c|} {sort.hpp}\\
\hline
void RadixSort(TVector< TKeyValuePair<TDate, unsigned long long> >\& vector)&Поразрядная сортировка.\\
\hline
\rowcolor{lightgray}
\multicolumn{2}{|c|} {vector.hpp}\\
\hline
class TVector&Динамический массив\\
\hline
\rowcolor{lightgray}
\multicolumn{2}{|c|} {pair.hpp}\\
\hline
struct TKeyValuePair&Пара ключ-значение\\
\hline
\rowcolor{lightgray}
\multicolumn{2}{|c|} {date.hpp}\\
\hline
class TDate&Дата с поразрадным представлением\\
\hline
\end{longtable}

$main.cpp$
\begin{lstlisting}[language=c++]
#include <iostream>
#include <fstream>
#include <string>
	
#include "pair.hpp"
#include "date.hpp"
#include "vector.hpp"
#include "sort.hpp"
	
	
int main()
{
	unsigned long long value;

	std::string strBuf;
	
	TKeyValuePair<TDate, unsigned long long> pairBuf;    

	TVector< TKeyValuePair<TDate, unsigned long long> > vector;

	std::ifstream fin("input.txt");
	std::ofstream fout("output.txt");

	while (fin >> strBuf >> value) {
		TDate dateBuf(strBuf);
		pairBuf.key = dateBuf;
		pairBuf.value = value;
		
		vector.PushBack(pairBuf);
	}
	
	RadixSort(vector);

	for (int i = 0; i < vector.Size(); ++i) {
		fout << vector[i].key.GetDateStr() << '\t' << vector[i].value << '\n';
	}
	
	fin.close();
	fout.close();
	
	return 0;
}
\end{lstlisting}

$sort.hpp$
\begin{lstlisting}[language=c++]
#pragma once
const int COUNT_SIZE = 10;

void RadixSort(TVector< TKeyValuePair<TDate, unsigned long long> >& vector) {
	int prefSumCountArray[COUNT_SIZE];  // index for next element with this digit at the place
	TVector< TKeyValuePair<TDate, unsigned long long> > tmpVector(vector.Size());

	// key[] = {y, y, y, y, m, m, d, d}
	for (int digitOffset = DATE_SIZE - 1; digitOffset >= 0; --digitOffset) {

		for (int i = 0; i < COUNT_SIZE; ++i) {      // init auxiliary arrays with zeros
			prefSumCountArray[i] = 0;
		}
		for (int i = 0; i < vector.Size(); ++i) {   // Counting keys with same digit
			++prefSumCountArray[vector[i].key[digitOffset] - '0'];
		}
		for (int i = 1; i < COUNT_SIZE; ++i) {      // Prefix sum for index of beginning of a series of same keys
			prefSumCountArray[i] += prefSumCountArray[i - 1];
		}
		for (int i = vector.Size() - 1; i >= 0; --i) {   // Forming result
			tmpVector[prefSumCountArray[vector[i].key[digitOffset] - '0'] - 1] = vector[i];
			--prefSumCountArray[vector[i].key[digitOffset] - '0'];
		}
		for (int i = 0; i < vector.Size(); ++i) {   // Result
			vector[i] = tmpVector[i];
		}
	}
}
\end{lstlisting}

$vector.hpp$
\begin{lstlisting}[language=c++]
template<typename T>
class TVector
{
private:
	int size;
	int capacity; // max_length before reallocate
	T* dataPtr;

public:
	// Constructors
	TVector();
	TVector(const int size);
	TVector(const int size, T initValue);

	TVector(const TVector& other);
	TVector& operator =(const TVector& other);

	TVector(TVector&& other);
	TVector& operator =(TVector&& other);

	// Destructor
	~TVector();

	// 
	int Size(); // Returns length
	bool Empty(); // length == 0
	T& operator [](const int index);
	void PushBack(const T& element); // Returns 0 in case of failure
	T PopBack(); // Removes last element and returns it

};
\end{lstlisting}

$pair.hpp$
\begin{lstlisting}[language=c++]
template<typename TKey, typename TValue>
struct TKeyValuePair
{
	TKey key;
	TValue value;

	TKeyValuePair() = default;

	bool operator <(const TKeyValuePair& other) const;
};
\end{lstlisting}

$date.hpp$
\begin{lstlisting}[language=c++]
class TDate
{
private:
	std::string strDate;
	char digits[DATE_SIZE];
public:
	TDate() = default;
	TDate(const std::string& str);
	char operator [](int index);
	bool operator <(TDate& other);

	std::string GetDateStr() const;
};
\end{lstlisting}

\pagebreak

\section{Консоль}
\begin{alltt}
root@77eb62ed9309:/workspaces/DA-Laboratory-Work/Lab_n1/solution# g++ -std=c++17 -Wall -Wextra -I. -o solution main.cpp
root@77eb62ed9309:/workspaces/DA-Laboratory-Work/Lab_n1/solution# cat input.txt
1.1.1   13207862122685464576
01.02.2008      7670388314707853312
1.1.1   4588010303972900864
01.02.2008      12992997081104908288
root@77eb62ed9309:/workspaces/DA-Laboratory-Work/Lab_n1/solution# ./solution 
root@77eb62ed9309:/workspaces/DA-Laboratory-Work/Lab_n1/solution# cat output.txt 
1.1.1   13207862122685464576
1.1.1   4588010303972900864
01.02.2008      7670388314707853312
01.02.2008      12992997081104908288
\end{alltt}
\pagebreak

